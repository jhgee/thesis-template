%%% File encoding is UTF-8
% !TEX root = ../MainFile.tex

\begin{appendix}
\printbibliography
\chapter{Anhang}
 \section{Zusätzliche Abbildungen}
 Im folgenden Kapitel sind Abbildungen mit Kurven aus Datenblättern und
 zusätzlichen Graphen zu sehen, auf die im entsprechenden Abschnitt der Arbeit
 jeweils verwiesen worden ist.

\begin{figure}
  \includegraphics[height=8cm]{example-image-a}
  \caption{Beispielbild}
\end{figure}

\begin{figure}
  \includegraphics[height=5cm]{example-image-b}
  \caption{Beispielbild}
\end{figure}

\begin{figure}
  \includegraphics[height=7cm]{example-image-c}
  \caption{Beispielbild}
\end{figure}

 \FloatBarrier
 \newpage
\section{Abkürzungen und Notation}
\printglossary[type=\acronymtype,title=Abkürzungen]
\printglossary[title=Notation]

%%% File encoding is UTF-8
% !TEX root = ../MainFile.tex

\chapter*{Danksagung}
\label{sec:acknowledgements}
\noindent
Ein Dankeschön möchte ich Prof. Dr. Witz
aussprechen für die Betreuung dieser Arbeit.

Bei Prof. Dr. Kowitz bedanke ich mich für die Zusage als Korreferent.

\Blindtext


\end{appendix}

  \cleardoublepage\thispagestyle{empty}
  \null\vfill
  \noindent
  \begin{tabular}[t]{p{0.175\textwidth}p{0.825\textwidth-4\tabcolsep}}
    \bfseries\large Erkl\"arung&nach
    \S13(8)%
    der Prüfungsordnung für den
    Bachelor-Studiengang Physik und den Master-Studiengang Physik
    an der Universität Göttingen:\\[1em]
    &Hiermit erkläre ich, dass ich diese Abschlussarbeit
    selbständig verfasst habe, keine anderen als die
    angegebenen Quellen und Hilfsmittel benutzt habe und alle Stellen,
    die wörtlich oder sinngemä\ss{} aus veröffentlichten Schriften
    entnommen wurden, als solche kenntlich gemacht habe.

    Darüberhinaus erkläre ich, dass diese Abschlussarbeit nicht, auch nicht
    auszugsweise, im Rahmen einer nichtbestandenen Prüfung an dieser oder
    einer anderen Hochschule eingereicht wurde.\\[1em]
    &\begin{center}Göttingen, den 11. November 2011\end{center}\\[1.5cm]
    &\begin{center}(Max Mustermann)\end{center}
  \end{tabular}
  \vfill
