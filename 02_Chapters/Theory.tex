%%% File encoding is UTF-8
% !TEX root = ../MainFile.tex

\chapter{Grundlagen}
\label{sec:theory}
Das klassische Problem der schiefen Ebenen mit Winkel \gls{angle} und Distanz
\gls{distance} ist sehr spannend und immer wieder nett zu rechnen.

\begin{equation}
  \gls{distance}_\perp = \gls{distance} \, \sin{\gls{angle}}
  \label{eq:perpendicular}
\end{equation}

Die senkrechte Distanzkomponente ergibt sich durch einfache Trigonometrie
\Eqref{eq:perpendicular}

\begin{equation}
  E=m c^2
  \label{eq:longTongue}
\end{equation}
Die wichtigsten Formeln sollten in keiner Arbeit fehlen \Eqref{eq:longTongue}.

\Blindtext[4][2]
